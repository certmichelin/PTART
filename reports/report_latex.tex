%-----------------------------------------------------------------%
%         Template based on cyber@cfreg LaTeX template            %
%                                                                 %
% https://github.com/cyber-cfreg/Penetration-Test-Report-Template %
%-----------------------------------------------------------------%

\documentclass[12pt, table,dvipsnames]{article}

%-----------------------------------------------------%
% Package import                                      %
%-----------------------------------------------------%
\usepackage[export]{adjustbox}
\usepackage{afterpage}
\usepackage{amsmath}
\usepackage{amssymb}
\usepackage{appendix}
\usepackage[USenglish]{babel}
\usepackage[style=numeric,backend=biber, sorting=none]{biblatex}
\usepackage{booktabs}
\usepackage{calc}
\usepackage{capt-of}
\usepackage[tableposition=top]{caption}
\usepackage{color}
\usepackage{csquotes}
\usepackage{easyfig}
\usepackage{etoolbox}
\usepackage{fancyhdr}
\usepackage{fancyvrb}
\usepackage[T1]{fontenc}
\usepackage{float}
\usepackage{framed}
\usepackage[letterpaper,left=2.5cm,right=2.5cm,top=2.5cm,bottom=2.5cm,headheight=15pt]{geometry}
\usepackage{graphicx}
\usepackage{hyperref}
\usepackage{iftex}
\usepackage{ifthen}
\usepackage[utf8]{inputenc}
\usepackage{lipsum} 
\usepackage{listings}
\usepackage{lmodern}
\usepackage{longtable}
\usepackage{lscape}
\usepackage{makecell}
\usepackage{mathptmx}
\usepackage[parfill]{parskip}
\usepackage[final]{pdfpages}
\usepackage{pgf-pie}
\usepackage{pgfplots}   
\usepackage{placeins}
\usepackage{rotating}
\usepackage{tabularx}
\usepackage{tabulary}
\usepackage[most]{tcolorbox}
\usepackage{textcomp}
\usepackage{tikz}
\usepackage{titlesec}
\usepackage{tocloft}
\usepackage{wrapfig}
\usepackage{xcolor}
\usepackage{xurl}
%-----------------------------------------------------%
% End of Package import                               %
%-----------------------------------------------------%

%-----------------------------------------------------%
% Aliases                                             %
%-----------------------------------------------------%
\newcommand{\doctitle}{Pentest Report}
\newcommand{\project}{\VAR{project.name | escape}}
\newcommand{\client}{\VAR{project.client | escape}}
\newcommand{\teamname}{Michelin CERT}
\newcommand{\company}{Michelin}
\newcommand{\classification}{CONFIDENTIAL}
\newcommand{\logopath}{resources/logo.png}
\newcommand{\companylogo}{resources/companylogo.png}
%-----------------------------------------------------%
% End of Aliases                                      %
%-----------------------------------------------------%

%-----------------------------------------------------%
% Docstyle                                            %
%-----------------------------------------------------%
\BLOCK{ if project.client }   
	\lhead{\project\ - \client}
\BLOCK{ else } 
	\lhead{\project}
\BLOCK{ endif } 
\rhead{\teamname}
\lfoot{\today}
\cfoot{\color{red}{\classification}}
\rfoot{Page \thepage}

\floatstyle{plaintop}
\restylefloat{table}

%Add paragraph
\titleformat{\paragraph}
{\normalfont\normalsize\bfseries}{\theparagraph}{1em}{}
\titlespacing*{\paragraph}
{0pt}{3.25ex plus 1ex minus .2ex}{1.5ex plus .2ex}

% For dotted table of content
\makeatletter
\renewcommand*\l@section{\@dottedtocline{1}{1.5em}{2.3em}}
\renewcommand*\l@subsection{\@dottedtocline{2}{2.3em}{3.2 em}}
\makeatother

%more control over table layouts
\newcolumntype{R}{>{\raggedleft\arraybackslash}X}
\restylefloat{table}

% Horizontal line and separator
\newcommand{\HRule}{\rule{\linewidth}{0.5mm}}
\newcommand{\hseparator}{\hspace{0.3em}-\hspace{0.3em}}

\fancypagestyle{footmark}{
	\ps@@fancy % use {fancy} style as a base of {footmark}
	\fancyfoot[C]{\footmark}
}

% sets value of \footmark and sets the correct style for this page only
\newcommand\markfoot[1]{\gdef\footmark{#1}\thispagestyle{footmark}}

% Page layout
\pagestyle{fancy}

\addbibresource{references.bib}
\hypersetup{colorlinks, allcolors=black}

%badge
\newtcbox{\badge}[1][red]{
	on line, 
	arc=2pt,
	colback=#1!50!black,
	colframe=#1!50!black,
	fontupper=\color{white},
	boxrule=1pt, 
	boxsep=0pt,
	left=4pt,
	right=4pt,
	top=2pt,
	bottom=2pt
}

%List of figures
\makeatletter
\renewcommand\listoftables{%
	\@starttoc{lot}%
}
\makeatother

%Colors
\definecolor{bc-badge-p1}{HTML}{d13535}
\definecolor{bc-badge-p2}{HTML}{ff6900}
\definecolor{bc-badge-p3}{HTML}{f0ad4e}
\definecolor{bc-badge-p4}{HTML}{5eae00}
\definecolor{bc-badge-p5}{HTML}{0278b8}

\definecolor{cvss-badge-dark}{HTML}{343a40}
\definecolor{cvss-badge-danger}{HTML}{dc3545}
\definecolor{cvss-badge-warning}{HTML}{ffc107}
\definecolor{cvss-badge-success}{HTML}{28a745}
\definecolor{cvss-badge-secondary}{HTML}{6c757d}

\definecolor{fix-complexity-0}{HTML}{6c757d}
\definecolor{fix-complexity-1}{HTML}{dc3545}
\definecolor{fix-complexity-2}{HTML}{ffc107}
\definecolor{fix-complexity-3}{HTML}{28a745}

\BLOCK{for label in labels }
\definecolor{label-color-\VAR{label.id }}{HTML}{\VAR{label.color | replace("#","")}}
\BLOCK{ endfor }

% For code highlight
\newcommand{\VerbBar}{|}
\newcommand{\VERB}{\Verb[commandchars=\\\{\}]}
\DefineVerbatimEnvironment{Highlighting}{Verbatim}{commandchars=\\\{\}}
% Add ',fontsize=\small' for more characters per line
\definecolor{shadecolor}{RGB}{248,248,248}
\newenvironment{Shaded}{\begin{snugshade}}{\end{snugshade}}
\newcommand{\AlertTok}[1]{\textcolor[rgb]{0.94,0.16,0.16}{#1}}
\newcommand{\AnnotationTok}[1]{\textcolor[rgb]{0.56,0.35,0.01}{\textbf{\textit{#1}}}}
\newcommand{\AttributeTok}[1]{\textcolor[rgb]{0.77,0.63,0.00}{#1}}
\newcommand{\BaseNTok}[1]{\textcolor[rgb]{0.00,0.00,0.81}{#1}}
\newcommand{\BuiltInTok}[1]{#1}
\newcommand{\CharTok}[1]{\textcolor[rgb]{0.31,0.60,0.02}{#1}}
\newcommand{\CommentTok}[1]{\textcolor[rgb]{0.56,0.35,0.01}{\textit{#1}}}
\newcommand{\CommentVarTok}[1]{\textcolor[rgb]{0.56,0.35,0.01}{\textbf{\textit{#1}}}}
\newcommand{\ConstantTok}[1]{\textcolor[rgb]{0.00,0.00,0.00}{#1}}
\newcommand{\ControlFlowTok}[1]{\textcolor[rgb]{0.13,0.29,0.53}{\textbf{#1}}}
\newcommand{\DataTypeTok}[1]{\textcolor[rgb]{0.13,0.29,0.53}{#1}}
\newcommand{\DecValTok}[1]{\textcolor[rgb]{0.00,0.00,0.81}{#1}}
\newcommand{\DocumentationTok}[1]{\textcolor[rgb]{0.56,0.35,0.01}{\textbf{\textit{#1}}}}
\newcommand{\ErrorTok}[1]{\textcolor[rgb]{0.64,0.00,0.00}{\textbf{#1}}}
\newcommand{\ExtensionTok}[1]{#1}
\newcommand{\FloatTok}[1]{\textcolor[rgb]{0.00,0.00,0.81}{#1}}
\newcommand{\FunctionTok}[1]{\textcolor[rgb]{0.00,0.00,0.00}{#1}}
\newcommand{\ImportTok}[1]{#1}
\newcommand{\InformationTok}[1]{\textcolor[rgb]{0.56,0.35,0.01}{\textbf{\textit{#1}}}}
\newcommand{\KeywordTok}[1]{\textcolor[rgb]{0.13,0.29,0.53}{\textbf{#1}}}
\newcommand{\NormalTok}[1]{#1}
\newcommand{\OperatorTok}[1]{\textcolor[rgb]{0.81,0.36,0.00}{\textbf{#1}}}
\newcommand{\OtherTok}[1]{\textcolor[rgb]{0.56,0.35,0.01}{#1}}
\newcommand{\PreprocessorTok}[1]{\textcolor[rgb]{0.56,0.35,0.01}{\textit{#1}}}
\newcommand{\RegionMarkerTok}[1]{#1}
\newcommand{\SpecialCharTok}[1]{\textcolor[rgb]{0.00,0.00,0.00}{#1}}
\newcommand{\SpecialStringTok}[1]{\textcolor[rgb]{0.31,0.60,0.02}{#1}}
\newcommand{\StringTok}[1]{\textcolor[rgb]{0.31,0.60,0.02}{#1}}
\newcommand{\VariableTok}[1]{\textcolor[rgb]{0.00,0.00,0.00}{#1}}
\newcommand{\VerbatimStringTok}[1]{\textcolor[rgb]{0.31,0.60,0.02}{#1}}
\newcommand{\WarningTok}[1]{\textcolor[rgb]{0.56,0.35,0.01}{\textbf{\textit{#1}}}}

\makeatletter
\patchcmd\longtable{\par}{\if@noskipsec\mbox{}\fi\par}{}{}
\makeatother

\providecommand{\tightlist}{%
	\setlength{\itemsep}{0pt}\setlength{\parskip}{0pt}}

%-----------------------------------------------------%
% End of Docstyle                                     %
%-----------------------------------------------------%                                          
%-----------------------------------------------------%
% Begin Document                                      %
%-----------------------------------------------------%
\begin{document}
\title{\doctitle}
\pagenumbering{arabic}

%-----------------------------------------------------%
% Title Page                                          %
%-----------------------------------------------------%
\begin{titlepage}
	\begin{center}
		
		\vspace{0.25cm}
		\begin{figure}[H]
			\centering
			\includegraphics[width=\textwidth/3]{\logopath}
		\end{figure}
		
		% Title
		\HRule\\[0.2cm]
		{\huge\bfseries\doctitle : \\ \project}
		\HRule\\[0.5cm]
		\vfill
		% End of Title

		% Client Name
		\BLOCK{ if project.client }
			\Large{Client:}
			\LARGE{\textbf{\client}}
			\vfill   
		\BLOCK{ endif } 
		% End of Client name
		
		% Team Name
		\Large{Conducted by:}
		\LARGE{\textbf{\teamname}}
		\vfill
		% End of Team Name
		
		% Date
		\Large{From \VAR{ project.start_date } to \VAR{ project.end_date }}
		\vfill
		% End of Date
		
		% Pentesters.
		Auditor(s):
		\begin{center}
			\small{
			\begin{itemize}
				\centering
				\BLOCK{ for pentester in project.pentesters.all() }
					\item[] \VAR{ pentester.username | escape } - \VAR{ pentester.first_name | escape } \VAR{ pentester.last_name | escape }
				\BLOCK{ endfor }
			\end{itemize}
			}
		\end{center}
		\vfill
		% End of Pentesters.
		
		\includegraphics[width=\textwidth/4]{\companylogo}
				
		% % Bottom of the page
		{\footnotesize {\color{red}{\classification}}}
		
	\end{center}
\end{titlepage}
%-----------------------------------------------------%
% End of Title Page                                   %
%-----------------------------------------------------%

%-----------------------------------------------------%
% Table of Contents                                   %
%-----------------------------------------------------%
\renewcommand{\contentsname}{Table of Contents}
\tableofcontents\thispagestyle{fancy}
\newpage
%-----------------------------------------------------%
% End of Table of Contents                            %
%-----------------------------------------------------%

%-----------------------------------------------------%
% Executive Summary                                   %
%-----------------------------------------------------%
\section{Report Overview}
\subsection{Executive Summary}
\VAR{ project.executive_summary | mdtolatex }
\newpage
%-----------------------------------------------------%
% End of Executive Summary                            %
%-----------------------------------------------------%

%-----------------------------------------------------%
% Scope                                               %
%-----------------------------------------------------%
\subsection{Scope}
\paragraph{Engagement Overview}
% This is where you should briefly explain when the penetration test was performed and how the team came into this work. This should be followed by a list of goals for the engagement.
% List goals for the team in the engagement targeted at your client's executives (non-technical).
\VAR{ project.engagement_overview | mdtolatex }

\paragraph{Scope of Engagement}
%Automatic scope list
\begin{itemize}
	\BLOCK{ for line in project.scope.splitlines() }
		\item \VAR{ line | escape }
	\BLOCK{ endfor }
\end{itemize}
\newpage
%-----------------------------------------------------%
% End of Scope                                        %
%-----------------------------------------------------%
%-----------------------------------------------------%
% Methodologies                                         %
%-----------------------------------------------------%
\BLOCK{ if project.methodologies.all() and project.methodologies.all()|length > 0 }
	\section{Testing Methodology}
	 
	\BLOCK{ for methodology in project.methodologies.all() }
		\subsection{\VAR{ methodology.name | escape }}
		
		\VAR{ methodology.description | mdtolatex }
	\BLOCK{ endfor }

	\newpage
\BLOCK{ endif } 
%-----------------------------------------------------%
% End of Methodologies                                %
%-----------------------------------------------------%
%-----------------------------------------------------%
% Observation                                         %
%-----------------------------------------------------%
\section{Observations}

\BLOCK{ for assessment in project.assessment_set.all() }
	\BLOCK{ if assessment.displayable_hits }
		\paragraph{\VAR{assessment.name | escape }}

		\begin{table}[H]
			\centering
			{\setlength{\extrarowheight}{2pt}%
			\begin{tabularx}{\textwidth}{| l | l | X | l | l |}
					\BLOCK{ for hit in assessment.displayable_hits() }
					\hline 
					\badge[bc-badge-p\VAR{ hit.severity }]{P\VAR{ hit.severity }} & \VAR{ hit.get_unique_id()} & \VAR{hit.title | escape } & \BLOCK{ if hit.get_cvss_value() == "---" }\badge[cvss-badge-secondary]{ N.C }\BLOCK{ elif hit.get_cvss_value() <= 0 }\badge[cvss-badge-secondary]{\VAR{ hit.get_cvss_value()}}\BLOCK{ elif hit.get_cvss_value() <= 4.0 }\badge[cvss-badge-success]{\VAR{ hit.get_cvss_value()}}	\BLOCK{ elif hit.get_cvss_value() <= 7.0 }\badge[cvss-badge-warning]{\VAR{ hit.get_cvss_value()}}\BLOCK{ elif hit.get_cvss_value() <= 9.0 }	\badge[cvss-badge-danger]{\VAR{ hit.get_cvss_value()}}\BLOCK{ else }\badge[cvss-badge-dark]{\VAR{ hit.get_cvss_value()}}\BLOCK{ endif } & \badge[fix-complexity-\VAR{ hit.fix_complexity }]{\VAR{ hit.get_fix_complexity_str()}}\\
					\BLOCK{ endfor }
					\hline 	
			\end{tabularx}}
		\end{table}
	\BLOCK{ endif }
\BLOCK{ endfor }

\begin{figure}[H]
	\centering
\begin{tikzpicture}
	\begin{axis} [
		bar width         =20pt,
		axis y line       = none,
		axis lines*       = left,
		tickwidth         = 0pt,
		enlarge y limits  = 0.2,
		enlarge x limits  = 0.2,
		symbolic x coords = {P1, P2, P3, P4, P5},
		nodes near coords
		]
		\addplot [ybar,fill = black]  coordinates {(P1,\VAR{project.p1_hits()|length})};
		\addplot [ybar,fill = red]    coordinates {(P2,\VAR{project.p2_hits()|length})};
		\addplot [ybar,fill = yellow] coordinates {(P3,\VAR{project.p3_hits()|length})};
		\addplot [ybar,fill = green]  coordinates {(P4,\VAR{project.p4_hits()|length})};
		\addplot [ybar,fill = blue]   coordinates {(P5,\VAR{project.p5_hits()|length})};
	\end{axis}
\end{tikzpicture}
\caption{Statistics of vulnerabilities by severity}
\end{figure}

\begin{figure}[H]
    \centering
    \begin{tikzpicture}
        \pie
        [
            rotate=180, 
            after number=,
            /tikz/nodes={text=black, font=\normalfont},
            /tikz/every pin/.style={align=center, text=black, font=\normalfont},
            sum=auto,
			text = legend
        ]
        {
			\BLOCK{ for key, value in project.labels_statistics().items() }
				\BLOCK{ if loop.index == 1 }\VAR{ value|string + '/' + key }\BLOCK{  else }\VAR{ ',' + value|string + '/' + key }\BLOCK{ endif }  
			\BLOCK{ endfor }
	    }
    \end{tikzpicture}
    \caption{Statistics of vulnerabilities by label}
\end{figure}
\newpage
%-----------------------------------------------------%
% End of Observations                                 %
%-----------------------------------------------------%

%-----------------------------------------------------%
% Recommendations                                     %
%-----------------------------------------------------%
\BLOCK{ if project.recommendation_set.all() and project.recommendation_set.all()|length > 0 }
	\section{Recommendations}
	\BLOCK{ for recommendation in project.recommendation_set.all() }
		\subsection{\VAR{ recommendation.name | escape }}
		\VAR{ recommendation.body | mdtolatex }
	\BLOCK{ endfor } 
	\newpage
\BLOCK{ endif } 
%-----------------------------------------------------%
% End of Recommendations                              %
%-----------------------------------------------------%
%-----------------------------------------------------%
% Technical Findings                                  %
%-----------------------------------------------------%
\section{Technical Findings}

\BLOCK{ for assessment in project.assessment_set.all() }
	\BLOCK{ if assessment.displayable_hits }
		\subsection{\VAR{assessment.name | escape }}
		\BLOCK{ for hit in assessment.displayable_hits() }
			\subsubsection{\VAR{ hit.get_unique_id()} - \VAR{hit.title | escape }}

			\begin{table}[H]
				\centering
				{\setlength{\extrarowheight}{2pt}%
				\begin{tabular}{| l | p{12cm} |}
					\hline 
					Severity & \badge[bc-badge-p\VAR{ hit.severity }]{P\VAR{ hit.severity }}\\
					\hline
					\BLOCK{ if hit.get_cvss_value() == "---" }
						CVSS Base Score & \badge[cvss-badge-secondary]{ N.C } \\
					\BLOCK{ elif hit.get_cvss_value() <= 0 }
						CVSS Base Score & \badge[cvss-badge-secondary]{\VAR{ hit.get_cvss_value()} - None} (\VAR{ hit.get_cvss_string() }) \\
					\BLOCK{ elif hit.get_cvss_value() <= 4.0 }
						CVSS Base Score & \badge[cvss-badge-success]{\VAR{ hit.get_cvss_value()} - Low} (\VAR{ hit.get_cvss_string() }) \\
					\BLOCK{ elif hit.get_cvss_value() <= 7.0 }
						CVSS Base Score & \badge[cvss-badge-warning]{\VAR{ hit.get_cvss_value()} - Medium} (\VAR{ hit.get_cvss_string() }) \\
					\BLOCK{ elif hit.get_cvss_value() <= 9.0 }
						CVSS Base Score & \badge[cvss-badge-danger]{\VAR{ hit.get_cvss_value()} - High} (\VAR{ hit.get_cvss_string() }) \\
					\BLOCK{ else }
						CVSS Base Score & \badge[cvss-badge-dark]{\VAR{ hit.get_cvss_value()} - Critical} (\VAR{ hit.get_cvss_string() }) \\
					\BLOCK{ endif } 
					\hline 
					Fix Complexity & \badge[fix-complexity-\VAR{ hit.fix_complexity }]{\VAR{ hit.get_fix_complexity_str()}} \\
					\hline 
					Asset & \VAR{ hit.asset | escape  } \\
					\hline 
					Label(s) &  \BLOCK{ for label in hit.labels.all() } \badge[label-color-\VAR{ label.id }]{\VAR{ label.title }}\BLOCK{ endfor } \\	
					\hline 
					Reference(s) & \BLOCK{ if hit.hitreference_set.all() and hit.hitreference_set.all()|length > 0 } \begin{itemize} \BLOCK{ for reference in hit.hitreference_set.all() } \item \VAR{ reference.name | escape } - \VAR{ reference.url | escape } \BLOCK{ endfor } \end{itemize} \BLOCK{ endif } \\
					\hline 
				\end{tabular}}
			\end{table}

			\VAR{ hit.body | mdtolatex }
		
			\BLOCK{ for screenshot in hit.screenshot_set.all() }
				\begin{figure}[H]
					\centering
					\includegraphics[width=\textwidth]{./screenshots/\VAR{screenshot.pk}.png}
					\caption{\VAR{screenshot.caption | escape }}
				\end{figure}
			\BLOCK{ endfor }

			\BLOCK{ if hit.remediation }
				\paragraph{Remediation}
				\VAR{ hit.remediation | mdtolatex }
			\BLOCK{ endif }
		\BLOCK{ endfor }
		\newpage
	\BLOCK{ endif }
\BLOCK{ endfor }
%-----------------------------------------------------%
% End of Technical Findings                           %
%-----------------------------------------------------%

%-----------------------------------------------------%
% Attack Scenarios                                    %
%-----------------------------------------------------%
\section{Attack Scenarios}

\BLOCK{ for attackscenario in project.attackscenario_set.all() }
	\subsection{\VAR{attackscenario.name | escape }}
	\VAR{ attackscenario.body | mdtolatex }
    \begin{figure}[H]
		\centering
		\includegraphics[scale=0.5]{./attackscenarios/\VAR{attackscenario.id}.png}
		\caption{\VAR{attackscenario.name | escape }}
	\end{figure}


\BLOCK{ endfor }
%-----------------------------------------------------%
% End of Attack Scenarios                             %
%-----------------------------------------------------%

%-----------------------------------------------------%
% Conclusion                                          %
%-----------------------------------------------------%
\section{Conclusion}
\VAR{ project.conclusion | mdtolatex }
\newpage
%-----------------------------------------------------%
% End of Conclusion                                   %
%-----------------------------------------------------%

%-----------------------------------------------------%
% Appendices                                          %
%-----------------------------------------------------%
\appendix
\appendixpage
\addappheadtotoc

% This should be a network topology that represents the in scope network from the engagement
% The example below was made using Lucidchart (https://lucid.app):
\section{Network Topology}

\BLOCK{ for host in project.host_set.all() }
	\subsection{\VAR{ host.ip } - \VAR{host.hostname | escape }}

	\begin{table}[H]
		\centering
		{\setlength{\extrarowheight}{2pt}%
			\begin{tabular}{| l | p{12cm} |}
				\hline 
				OS & \VAR{host.os | escape }\\
				\hline 
		\end{tabular}}
	\end{table}

	\VAR{host.notes | mdtolatex}

	\BLOCK{ for service in host.service_set.all() }
		\subsubsection{Port \VAR{service.port} - \VAR{service.protocol | escape }}

		\begin{table}[H]
			\centering
			{\setlength{\extrarowheight}{2pt}%
				\begin{tabular}{| l | p{12cm} |}
					\hline 
					Name & \VAR{service.name | escape }\\
					\hline
					Version & \VAR{service.version | escape }\\
					\hline 
					Banner & \VAR{service.banner | escape }\\
					\hline  
			\end{tabular}}
		\end{table}
	\BLOCK{ endfor }
\BLOCK{ endfor }
\newpage

\BLOCK{ if project.tools.all() and project.tools.all()|length > 0 }
	\section{Tools}

	\begin{table}[H]
		\centering
		{\setlength{\extrarowheight}{2pt}%
		\begin{tabular}{| l | l |}
			\hline
			\BLOCK{ for tool in project.tools.all() }
				\VAR{ tool.name | escape } & \VAR{ tool.url | escape }\\
				\hline
			\BLOCK{ endfor } 
		\end{tabular}}
	\end{table}
	\newpage
\BLOCK{ endif } 


\BLOCK{ if project.methodologies.all() and project.methodologies.all()|length > 0 }
	\section{Testing Methodology Details}
	 
	\BLOCK{ for methodology in project.methodologies.all() }
		\subsection{\VAR{ methodology.name | escape }}
		\BLOCK{ for module in methodology.module_set.all()}
			\subsubsection{\VAR{ module.name | escape }}
			\VAR{ module.description | mdtolatex }

			\BLOCK{ for case in module.case_set.all()}
				\paragraph{\VAR{ case.name | escape }}
				\VAR{ case.description | mdtolatex }
				
				\url{\VAR{ case.reference }}
			\BLOCK{ endfor }
		\BLOCK{ endfor }
	\BLOCK{ endfor }

	\newpage
\BLOCK{ endif } 

% This should be a table of all tools utilized during the engagement.
% An example of what this may look like is included below:
%\section{Tools}
%\newpage

\section{List of figures}
\listoffigures
%-----------------------------------------------------%
% End of Appendices                                   %
%-----------------------------------------------------%

\end{document}